\documentclass[uplatex,dvipdfmx,ja=standard]{beamer}
\usepackage{amsmath,amssymb}
\usepackage{bm}
\usepackage{graphicx}
\usepackage{ascmac}
\usepackage{listings,jvlisting} 
\usepackage{url}
\usetheme{Darmstadt}
%
%
%%% Begin:
\setbeamertemplate{navigation symbols}{}
\setbeamertemplate{footline}[frame number]
\lstset{
  basicstyle={\ttfamily\small},
  keywordstyle={\color{blue}},
  commentstyle={\color{green}},
  stringstyle=\color{red},
  tabsize=1,
  breaklines=true, %折り返し
}
\begin{document}
%
%
\title[Presentation]{Parsec による簡易インタプリタの実装と構文設計(最終発表)} 
\institute[Waseda Univ.]{\normalsize 機械科学・航空宇宙学科 3年 1w192224 田久健人}
\date{\today}

\begin{frame}
\titlepage 
\end{frame}

\begin{frame}[fragile]{はじめに}
 \begin{verbatim}
 中間発表の続きです...

 今回は,パーサではなく評価寄りの話です
 
 \end{verbatim}

 進捗→
 \url{https://github.com/tkyawa/project_research_a}

\end{frame}

\begin{frame}{abstract}
Haskell のパーサコンビネータのParsec を用いた簡易なインタプリタの実装を通して,パーサコンビネータの雰囲気を学び,実際に構文の設計を行う.
\end{frame}

\begin{frame}{目次}
    \tableofcontents
\end{frame}

\setcounter{section}{3}

\section{関数実装に向けた下準備}
\subsection{おさらい}

\begin{frame}[fragile]{これから拡張していく文法}
AST$\to$ 

    \begin{verbatim}
data Statement = Seq [Statement]
          | If Expr Statement Statement
          | While Expr Statement  
          | Assign String Expr
          | Skip

data Expr =  Var String   
          |  Integer Integer 
          |  Bool Bool
          |  (Op) Expr Expr
    \end{verbatim}
\end{frame}

\begin{frame}[fragile]{これから拡張していく文法}
(実際の構文$\to$)
 
\begin{verbatim}

Statement ::= Statement; Statement; ..,; Statement
            | if(Expr)then{Statement}else{Statement}
            | while(Expr)do{Statement}
            | String := Expr
            | skip
    \end{verbatim}
\end{frame}

\begin{frame}[fragile]{これから拡張していく文法}
実行時の値を保存する環境(実行コンテキスト),型の情報
    \begin{verbatim}
type Env = Map String TypeEnv  

data TypeEnv = TypeInteger Integer 
             | TypeBool Bool
             deriving(Show)

    \end{verbatim}
\end{frame}

\begin{frame}[fragile]{関数実装に必要なもの}
    関数の構文$\to$
    \begin{verbatim}
AST:
    data Expr = ...
              | Func String Statement

(実際の構文:)
    Func :== func(arg){body}
    \end{verbatim}
関数は式だが中身が文$\to$値(式)を返す文が必要
\end{frame}

\begin{frame}[fragile]{Return文の実装}
    文を,(形式上は)値を返すようにする (evalStatementの型を統一)

    $\to$Null型の実装を考える
    \begin{verbatim}
    data TypeEnv = ... 
                 | Null
    \end{verbatim}
    return以外の文は,Nullを返すようにする

\end{frame}

\subsection{Return文の実装}

\begin{frame}[fragile]{Returnによるjump機能の実装①}
一般的なReturn文・・・jumpの機能がある

例:C言語の場合
    \begin{verbatim}
int hoge()
{
    int temp = 1;
    return temp;
    temp = 2;
}
    \end{verbatim}
hoge()を呼ぶと,1が返ってくる$\to$returnで文の評価が終了している
\end{frame}

\begin{frame}[fragile]{(補足)Returnによるjump以外での同等の表現}
    \begin{verbatim}
Rust:

  bodyの最後に式を書くことを許し,それが関数の返り値となる
  returnで処理を中断して値返すことも可能(jump)

OCaml(関数型言語一般):

  そもそも全部Expr

    \end{verbatim}
必ずしもreturnによるjump機能は必須ではない

今回は,命令型としてのシンプルさを求めているためreturnによるjumpを導入
\end{frame}

\begin{frame}[fragile]{Returnによるjump機能の実装②}
値を返すのはreturnだけ$\to$

Nullの場合と値が得られた場合でパターンマッチ
\begin{verbatim}

\end{verbatim}

$\to$Maybeモナドの恩恵が受けられそう!

\end{frame}

\begin{frame}[fragile]{Returnによるjump機能の実装③}
    \begin{verbatim}
Seqenceの表現:

evalStatement env (Seq []) = return Nothing   

evalStatement env (Seq (h:t)) 
 = do 
  car <- evalStatement env h
  maybe(evalStatement env (Seq t))(return . Just)car
    \end{verbatim}
    Nothingで評価続行,Justで評価停止($\to$なんか逆...)
\end{frame}


\begin{frame}[fragile]{デモ}
    \begin{verbatim}
>> x := 1; return x; x := 10
1
>> return x
1
    \end{verbatim}

\end{frame}

\begin{frame}{まとめ}
return文の実装により,関数の返り値の指定が可能になった
\end{frame}


\section{関数の実装}

\begin{frame}[fragile]{AST}
    構文的には,関数の宣言,関数適用が必要

    \begin{verbatim}
data Expr = ...
          | Func String Statement
          | Apply Expr Expr     
    \end{verbatim}
今回は,どちらもExprに組み込む
\end{frame}


\begin{frame}[fragile]{実際の構文}
    \begin{verbatim}
Func:
    func_name :== func(arg){body}

Apply:
    var :== func_name(param) 
\end{verbatim}

\end{frame}

\begin{frame}[fragile]{Parser}
FuncのParser・・・愚直に実装できる

ApplyのParser・・・愚直に実装すると...
    \begin{verbatim}
parseApply :: Parser Expr
parseApply = do func <- parseExpr 
                param <- parens parseExpr 
                return $ Apply func param 

→予約語等で縛っていないため失敗条件(?)が緩い,
 優先度の高い位置に置くと失敗せずに無限ループに
    \end{verbatim}
演算子として見る,優先度高め$\to$builldExpressionParserに組み込む
\end{frame}

\begin{frame}[fragile]{parseApply}
専用の演算子,予約語を使わすに実装したい
    \begin{verbatim}
parseExpr = buildExpressionParser
       [[postfix (parens parseExpr) Apply]
       ,[binary "^" Pow AssocRight]
       .
       .
       ]
       exprTerm 
  where
    .
    .
    postfix args fun = Postfix (flip fun <$> args)
    \end{verbatim}

$\to$ (引数)を後置演算子として使用する(微妙?)

\end{frame}

\begin{frame}[fragile]{Funcの評価}
    関数型のようなものを用意し,関数型を返す
    \begin{verbatim}
イメージ:

evalExpr env (Func arg body) 
    = return (関数型の値)
    \end{verbatim}
\end{frame}

\subsection{関数型の検討案①}

\begin{frame}[fragile]{関数型の実装案①}
    \begin{verbatim}
案1:

data TypeEnv = ... 
             | Closure Env String Statement
    \end{verbatim}
Closure内のEnvの入れ子でスコープを表現する
\end{frame}

\begin{frame}[fragile]{案①によるApplyの評価①}
バラして,入れて,中身を評価
    \begin{verbatim}
手順:

(1) func_name を評価 → 関数ならClosureが返ってくる

(2) もしClosure closureEnv arg body が返ってきたら...

    (i)ClosureEnv に(arg param)をdefineVar

    (ii)evalStatement closureEnv body
    \end{verbatim}
\end{frame}

\begin{frame}[fragile]{案①によるApplyの評価②}

    \begin{verbatim}
evalExpr env (Apply funcname param) = do
    func <- evalExpr env funcname
    case func of 
        Closure closureEnv arg body -> do
            value <- evalExpr env param  
(i)         newenv <- defineVar closureEnv arg value
(ii)        result <- evalStatement closureEnv body 
            maybe (return Null) return result        
            _ -> error "Error in func"
    \end{verbatim}
\end{frame}


\begin{frame}{関数定義,関数適用の際の環境の動き①}
    \centering 
    \includegraphics[scale=0.31]{func1.png}
\end{frame}

\begin{frame}{関数定義,関数適用の際の環境の動き②}
    \centering 
    \includegraphics[scale=0.31]{func2.png}
\end{frame}

\begin{frame}{関数定義,関数適用の際の環境の動き③}
    \centering 
    \includegraphics[scale=0.31]{func3.png}
\end{frame}


\begin{frame}{関数定義,関数適用の際の環境の動き③}
    \centering 
    \includegraphics[scale=0.31]{func5.png}
\end{frame}

\begin{frame}{案①の問題点}
    \begin{itemize}
        \item 関数内から外側のスコープには一切アクセスできない
        \item スコープが関数宣言時に確定している
    \end{itemize}
\end{frame}

\begin{frame}{まとめ}
環境を入れ子構造にすることにより関数実行時のスコープを表現したが,閉じた関数しか表現できないという問題点がある
\end{frame}

\section{再帰呼び出し実装の検討}

\begin{frame}[fragile]{再帰呼び出しの構文}
    (すごい雑に表すと)
    \begin{verbatim}
hoge := func(x){x := x + 1; return hoge(x)}
    \end{verbatim}

\end{frame}

\subsection{再帰呼び出し実装における問題点}

\begin{frame}{現状の問題点}
    \begin{itemize}
        \item 環境問題(案①だとガチガチすぎ)
    \end{itemize}
スコープを取る動作を,関数宣言時ではなく関数適用時にする必要がある
\end{frame}

\begin{frame}{(補足)未定義関数のApplyについて}
評価の順番を考えると納得

func宣言時,bodyは評価していない
\end{frame}

\begin{frame}[fragile]{環境問題解決の方針}
    スコープを,ただの連想リストの入れ子で表現していた
    \begin{verbatim}
    \end{verbatim}
    $\to$スタックっぽくする
\end{frame}

\begin{frame}[fragile]{環境問題の解決}
    内側に新しい空の環境を取っていた

    $\to$もとの環境を参照する形にする.
    \begin{verbatim}
evalExpr env (Func arg body) = 
    return (Closure env arg body)

(Closureの中身は案①のまま)
    \end{verbatim}

    これにより,関数が開けた

\end{frame}

\subsection{案②:スタックベースの環境}

\begin{frame}[fragile]{案②:環境をスタック化する}
    新たなEnvの定義:
    \begin{verbatim}
type Env = IORef [Map String (IORef TypeEnv)]
    \end{verbatim}
Envを,MapのListで定義

Listをスタックとして扱う
\end{frame}

\begin{frame}[fragile]{スタックにまつわるヘルパー関数:push}
    \begin{verbatim}
push :: Env -> String -> TypeEnv -> IO Env 
push env var val = do
    valRef <- newIORef val
    cons <- readIORef env
    newIORef (Data.Map.fromList [(var, valRef)]:cons)
    \end{verbatim}
pushは,Listの先頭に新たな要素を追加
\end{frame}


\begin{frame}[fragile]{スタックにまつわるヘルパー関数:pop}
    \begin{verbatim}
pop :: Env -> IO Env 
pop env = do
        garbage <- readIORef env
        case garbage of
            (h:t) -> do newIORef t
            [] -> error "stack error"
    \end{verbatim}
popは,Listのcarを破棄
\end{frame}

\begin{frame}[fragile]{その他のヘルパー関数の変更}
スタック(List)の頭から順に捜査していく

    \begin{verbatim}
例:getValの場合

getVal :: Env -> String -> IO TypeEnv
getVal envRef var = do
  envStack <- readIORef envRef
  case envStack of
    (h:t) -> do
      cdr <- newIORef t
      maybe (getVal cdr var) readIORef (lookup var h)
    [] -> return Null
    \end{verbatim}
データ捜査はListとしての性質を用いる
\end{frame}

\begin{frame}[fragile]{スタックを使用した関数適用の挙動}
    \begin{verbatim}
手順:

(1) func_name を評価→関数ならClosureが返ってくる 

(2) もしClosure closureEnv arg body が返ってきたら...

    (i)   ClosureEnv に,(arg param)を組み込んだ
          新たな実行コンテキストをpush 

    (ii)  evalStatement newEnv body

    (iii) 使い終わった実行コンテキストはpop
    \end{verbatim}
\end{frame}

\begin{frame}[fragile]{案②によるApplyの評価}
    \begin{verbatim}
evalExpr env (Apply funcname param) = do
    func <- evalExpr env funcname
    case func of
        Closure closureEnv arg body -> do
            value <- evalExpr env param
(i)         newenv <- push closureEnv arg value
(ii)        result <- evalStatement newenv body
(iii)       garbage <- pop newenv
            maybe (return Null) return result
            _ -> error "Error in func"
    \end{verbatim}
スコープの挙動はスタックベース
\end{frame}

\begin{frame}[fragile]{デモ}
    \begin{verbatim}
(>> a := 100)

>> hoge := func(x){a := 5; return a}

>> return hoge(1)

>> return a
    \end{verbatim}
aが定義される場所によるスコープの挙動の差
\end{frame}

\subsection{スタックベースのスコープの挙動}

\begin{frame}[fragile]{両ケースに共通する動作}
    \begin{verbatim}
(i)   引数の(識別子,値)を,実行コンテキストをとってpush

(iii) 使い終わった実行コンテキストをpop
    \end{verbatim}
$\to$ (ii)のevalStatementだけが異なる,特にdefineVal
    \begin{verbatim}
    \end{verbatim}

\url{https://github.com/tkyawa/project_research_a/blob/master/rec/Env.hs}
\end{frame}

\begin{frame}[fragile]{スコープの挙動①local変数の場合}
    \centering 
    \includegraphics[scale=0.31]{local1.png}
\end{frame}


\begin{frame}[fragile]{スコープの挙動①local変数の場合}
    \centering 
    \includegraphics[scale=0.31]{local2.png}
\end{frame}


\begin{frame}[fragile]{スコープの挙動①local変数の場合}
    \centering 
    \includegraphics[scale=0.31]{local3.png}
\end{frame}


\begin{frame}[fragile]{スコープの挙動①local変数の場合}
    \centering 
    \includegraphics[scale=0.31]{local4.png}
\end{frame}


\begin{frame}[fragile]{スコープの挙動①local変数の場合}
    \centering 
    \includegraphics[scale=0.31]{local5.png}
\end{frame}

\begin{frame}[fragile]{スコープの挙動①local変数の場合}
    \centering 
    \includegraphics[scale=0.31]{local6.png}
\end{frame}

\begin{frame}[fragile]{スコープの挙動①local変数の場合}
    \centering 
    \includegraphics[scale=0.31]{local7.png}
\end{frame}

\begin{frame}[fragile]{スコープの挙動①local変数の場合}
    \centering 
    \includegraphics[scale=0.31]{local8.png}
\end{frame}

\begin{frame}[fragile]{スコープの挙動①local変数の場合}
    \centering 
    \includegraphics[scale=0.31]{local9.png}
\end{frame}

\begin{frame}[fragile]{スコープの挙動①local変数の場合}
    \centering 
    \includegraphics[scale=0.31]{local10.png}
\end{frame}

\begin{frame}[fragile]{スコープの挙動①local変数の場合}
    \centering 
    \includegraphics[scale=0.31]{local11.png}
\end{frame}


\begin{frame}[fragile]{スコープの挙動②global変数の場合}
    \centering 
    \includegraphics[scale=0.31]{glo2.png}
\end{frame}

\begin{frame}[fragile]{スコープの挙動②global変数の場合}
    \centering 
    \includegraphics[scale=0.31]{glo3.png}
\end{frame}

\begin{frame}[fragile]{スコープの挙動②global変数の場合}
    \centering 
    \includegraphics[scale=0.31]{glo4.png}
\end{frame}

\begin{frame}[fragile]{スコープの挙動②global変数の場合}
    \centering 
    \includegraphics[scale=0.31]{glo5.png}
\end{frame}

\begin{frame}[fragile]{スコープの挙動②global変数の場合}
    \centering 
    \includegraphics[scale=0.31]{glo6.png}
\end{frame}

\begin{frame}[fragile]{スコープの挙動②global変数の場合}
    \centering 
    \includegraphics[scale=0.31]{glo7.png}
\end{frame}

\begin{frame}[fragile]{スコープの挙動②global変数の場合}
    \centering 
    \includegraphics[scale=0.31]{glo8.png}
\end{frame}

\begin{frame}{まとめ}
環境をスタックで実装したことにより,スコープの表現と柔軟性が両立し,再帰呼び出しが行えるようになった
\end{frame}

\section{おまけ}

\begin{frame}{更なる拡張を考える}
    \begin{itemize}
        \item 多引数の関数
        \item 文法の整理
        \item Listの実装
        \item アクセス修飾子の実装
    \end{itemize}
\end{frame}

\begin{frame}[fragile]{多引数の関数}
    カリー化で実現可能
    \begin{verbatim}
例:add
    add := func(x){return func(y){return x + y}}
    \end{verbatim}
    実際に構文に組み込む際は,これのシュガーとして実装
\end{frame}

\begin{frame}{文法の整理}
二項演算等をプリミティブな関数として用意すれば,Exprに無理に組み込む必要がなくなる?

もしくはzero,succのみ用意して組んでいく?
\end{frame}

\begin{frame}[fragile]{Listの実装}
List型,cons演算子等をプリミティブに用意すれば,再帰呼び出しが可能なのでList内の要素へのアクセスも可能

\end{frame}

\begin{frame}[fragile]{アクセス修飾子の実装}
グローバルなスコープがスタックの底だとわかっているから,それを元に実装可能?
\end{frame}

\section{感想,その他}

\begin{frame}{まとめ,感想}
実装を通して,言語設計における悩みどころが体感できた
    \begin{itemize}
        \item 何を式,何を文とするか
        \item 各シンボルの導入,構文的な役割(; や{}等)
        \item 関数実装周りのあれこれ(構文的な位置付け,return,環境,etc...)
    \end{itemize}
\end{frame}


\begin{frame}{参考文献}
    [1] Thorsten Ball,設樂 洋爾, Go言語でつくるインタプリタ,オライリージャパン,2018年

    [2] Daniel P.Friedman,Matthias Felleisen,元吉 文男,横山 晶一,Scheme手習い,オーム社,2010年
    
    [3] \url{https://en.wikibooks.org/wiki/Write_Yourself_a_Scheme_in_48_Hours}

    最終アクセス:2021年8月1日
\end{frame}

%
\end{document}

%%% End
