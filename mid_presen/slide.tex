\documentclass[uplatex,dvipdfmx,ja=standard]{beamer}
\usepackage{amsmath,amssymb}
\usepackage{bm}
\usepackage{graphicx}
\usepackage{ascmac}
\usepackage{listings,jvlisting} 
\usetheme{Darmstadt}
%
%
%%% Begin:
\lstset{
  language={Haskell}, 
  basicstyle={\ttfamily\small},
  keywordstyle={\color{blue}},
  commentstyle={\color{green}},
  stringstyle=\color{red},
  tabsize=1,
  breaklines=true, %折り返し
}
\begin{document}
%
%
\title[Midterm Presentation]{Parsec による簡易インタプリタの実装と構文の設計} 
\institute[Waseda Univ.]{\normalsize 機械科学・航空宇宙学科 3年 1w192224 田久健人}
\date{\today}

\begin{frame}
\titlepage 
\end{frame}

\begin{frame}{abstract}
    Haskell のパーサコンビネータのParsec を用いた簡易なインタプリタの実装を通して,パーサコンビネータの雰囲気を学び,実際に構文の設計を行う.
\end{frame}

\begin{frame}{目次}
    \tableofcontents
\end{frame}

\section{モナディックパーサ概要}
\subsection{Monadの概要}
\begin{frame}[fragile]{Monadとは}
    (Haskellの型クラスとしての)Monadとは・・・
    \begin{lstlisting}[]
    class Monad m where 
    (>>=) :: m a -> (a -> m b) -> m b 
    return :: a -> m a
    \end{lstlisting}
\end{frame}

\begin{frame}{Monadとは}
    Monadの利点・・・
    \begin{itemize}
        \item \textbf{状態量}の扱いを,副作用を気にせずに行える場合がある
        \item 関数合成等ができる
    \end{itemize}
\end{frame}

\begin{frame}[fragile]{状態量について}
    状態量の例・・・
    \begin{verbatim}
    
    add: x -> e:e + 1

    \end{verbatim}

    関数呼び出しの度に異なる結果が返る
\end{frame}


\begin{frame}{状態量について}
    \begin{block}{定義1.1}
    ある値xが状態量を持つことを$T(x)$と表記する
    \end{block}
\end{frame}

\begin{frame}{状態量を用いたMonad概要図①}
    $\to$関数合成できない
\end{frame}

\begin{frame}{状態量を用いたMonad概要図②}
    bind,returnによって関数合成(っぽいこと)ができる
\end{frame}

\begin{frame}[fragile]{Monadの例1:Maybe ①}
    \begin{verbatim}
data Maybe a = Nothing
             | Just a 
            deriving (Eq, Ord)
    
instance Monad Maybe where
    (Just x) >>= f = f x
    Nothing >>= f = Nothing
    return x = Just x
    \end{verbatim}
    bindの実装が,JustとNothingのパターンマッチ
\end{frame}

\begin{frame}[fragile]{Monadの例1:Maybe ②}
    モナド値が取りうる全ての状態に対しパターンマッチを行うことにより,bindが実装されている
    \begin{verbatim}

    \end{verbatim}
    $\to$抽象的な概念と実装の橋渡し
\end{frame}

\begin{frame}[fragile]{Monadの例2:State}
    \begin{verbatim}
instance Monad (State s) where
    f >>= m = State $ \s ->
        let (s', a) = runState f s
            in runState (m a) s'

    \end{verbatim}
    $\to$取り出した値の状態に対するパターンマッチでbindを実装
\end{frame}

\begin{frame}[fragile]{Parser Monad ①}
    パーサをMonadとして扱うことを考える

    $\to$
\end{frame}

\begin{frame}[fragile]{Parser Monad ②}
    \begin{verbatim}
data Parser a = Parser (String -> [(a,String)])

instance Monad Parser where
    return a = Parser $ \cs -> [(a,cs)]
    p >>= f  = 
    Parser $ \cs -> let l = parse p cs
                       ll = map func l
                          in concat ll
        where
            func (a,cs') = parse (f a) cs'
    \end{verbatim}
\end{frame}

\subsection{Parsec の各パーサについて}

\begin{frame}[fragile]{Parsec の各パーサについて}
    \begin{verbatim}
    Parser :: String -> AST
    \end{verbatim}
\end{frame}


\begin{frame}[fragile]{最小構成のパーサ}
一文字を受理するパーサ
    \begin{verbatim}
mono :: Parser Char
mono = Parser $ \cs -> case cs of
                         ""     -> []
                         (c:cs') -> [(c,cs')]
    \end{verbatim}
必ず失敗するパーサ
    \begin{verbatim}
failure :: Parser a
failure = \inp -> []
    \end{verbatim}
\end{frame}


\begin{frame}{最小構成のパーサ}
    各パーサは,bindにより合成可能

\end{frame}

\subsection{四則演算のインタプリタ}

\begin{frame}[fragile]{四則演算の実装}
    全体の構成:
    \begin{verbatim}
    AST: data型で表現
    Lexer,Parser: Parsec の各関数で実装
    Eval: Haskellに渡す

    data AST = ...
    (lexer :: String -> Parser String)
    parser :: Parser AST
    eval :: AST -> Integer
    \end{verbatim}
\end{frame}

\begin{frame}[fragile]{演算子順位の実装}
    演算子順位は,パーサ実装で(おそらく)最初に頭を使う(面白い,難しい)部分

    Parsecでは,buildExpressionParser を用いて直感的(?)に実装可能
\end{frame}

\begin{frame}[fragile]{buildExpressionParser ①}
    実装例:
    \begin{verbatim}
expr :: Parser Expr
expr = buildExpressionParser
       [[binary "^" Pow AssocRight]
       ,[binary "*" Mul AssocLeft, 
        binary "/" Div AssocLeft]
       ,[binary "+" Add AssocLeft, 
        binary "-" Sub AssocLeft, prefix "-" Negate]
       ]
       atom
  where
    binary name fun assoc = 
        Infix (reservedOp name >> return fun) assoc
    prefix name fun = Prefix (fun <$ reservedOp name)
    \end{verbatim}

\end{frame}

\begin{frame}[fragile]{buildExpressionParser ②}
    つづき:
    \begin{verbatim}
atom :: Parser Expr
atom = do symbol "("
      x <- expr
      symbol ")"
      return x
   <|> (Const <$> natural)
    \end{verbatim}
\end{frame}

\begin{frame}{デモ①}
\end{frame}

\subsection{}

\begin{frame}{まとめ}
    Parser Monad
\end{frame}

\section{Parsec によるIMP の実装}
\subsection{実装}

\begin{frame}[fragile]{IMPの構文}
\end{frame}

\begin{frame}[fragile]{全体の構成}
    ファイル構成:
    \begin{verbatim}
    IMP - - Syntax.hs
        | - Parser.hs 
        | - Env.hs
        | - Eval.hs
        | - Main.hs
    \end{verbatim}
\end{frame}

\begin{frame}[fragile]{Syntax.hs}
    構文をASTで,ASTをデータ型で表現
    \begin{verbatim}
    data Command = Seq [Command]
          | If BExpr Command Command
          | While BExpr Command 
          | Assign String AExpr
          | Skip
          deriving (Eq,Show)

    data BExpr = Bool Bool
          |  Greater AExpr AExpr
          |  Less AExpr AExpr
          deriving (Eq,Show)
    \end{verbatim}
\end{frame}

\begin{frame}[fragile]{Syntax.hs}
    つづき
    \begin{verbatim}
    data AExpr = Id String 
          | Integer Integer
          | Add AExpr AExpr
          | Sub AExpr AExpr
          | Mul AExpr AExpr
          | Div AExpr AExpr
          | Pow AExpr AExpr
          | Fact AExpr
          | Negate AExpr
          deriving (Eq, Show)
    \end{verbatim}
\end{frame}

\begin{frame}[fragile]{Parser.hs}
    再帰的に表現
\end{frame}

\begin{frame}[fragile]{parseAExpr}
    四則演算のパーサと同じ雰囲気で実装
\end{frame}

\begin{frame}[fragile]{parseCommand}
    予約語を受理するパーサと式のパーサを組み合わせる

    例: parseIfCommand
    \begin{verbatim}
parseIfCommand :: Parser Command
parseIfCommand =
    do reserved "if"
       cond <- parseBExpr 
       reserved "then"
       stmt1 <- parseCommand
       reserved "else"
       If cond stmt1 <$> parseCommand
   \end{verbatim}
\end{frame}

\begin{frame}[fragile]{Seqについて}
    \begin{verbatim}
sequenceOfCommand =
    do list <- sepBy1 parseCommand semi
    return $ if length list == 1 
        then head list else Seq list
   \end{verbatim}

セミコロンをセパレータとして使用

\end{frame}

\begin{frame}[fragile]{(補足)セミコロンについて①}
    本実装におけるセミコロン:
   \begin{verbatim}
   Statement ; Statement ; ... ; Statement
   \end{verbatim}
   \begin{itemize}
       \item 文末には付けない
       \item Pascal等が同じ方式
   \end{itemize}
\end{frame}


\begin{frame}[fragile]{(補足.)セミコロンについて②}
    C言語のセミコロン
\end{frame}

\begin{frame}[fragile]{Env.hs}
    変数をData.map で実装
   \begin{verbatim}
   type Env = Map String Integer 
    \end{verbatim}
\end{frame}

\begin{frame}[fragile]{Eval.hs}
    Command(文)はEnvを返す
   \begin{verbatim}
evalCommand :: Env -> Command -> Env
    \end{verbatim}
    AExpr, BExpr(式)は値を返す
    \begin{verbatim}
evalBExpr :: Env -> BExpr -> Bool

evalAExpr :: Env -> AExpr -> Integer
    \end{verbatim}
\end{frame}

\begin{frame}[fragile]{デモ②}
    \begin{verbatim}
x := 1; 

x := 1; y := 2; if(x > y) then{x := 10} else{y := 10}

x := 1; while(x < 10) do {x := x + 1}
    \end{verbatim}
\end{frame}

\subsection{IMPの構文上の問題点}

\begin{frame}{IMPの構文上の問題点}
    IMPの型(IntegerとBoolean)は,構文上完全に切り離されている.
    (AExprとBExpr)
\end{frame}

\subsection{}

\begin{frame}{まとめ}
    \begin{itemize}
        \item \large IMPを実装した
        \item \large IMPの型表現の窮屈さを感じた
    \end{itemize}
\end{frame}

\section{IMPの改良}

\begin{frame}[fragile]{式を統合する}
    2つのExprを統合する
    \begin{verbatim}
    data Expr =  Var String   
          |  Integer Integer 
          |  Bool Bool
          |  Negative Expr
          |  Add Expr Expr
          ...
          |  Sub Expr Expr
          |  Greater Expr Expr
          |  Less Expr Expr
          |  Equal Expr Expr
          deriving (Eq,Show)
    \end{verbatim}

\end{frame}

\begin{frame}[fragile]{型の表現}
    Exprの統合により,これまでAExpr.Integerに限定していた変数の型をExprの各リテラル(Integer,Bool)に対応できるようにする必要がある.

\end{frame}

\begin{frame}[fragile]{Env.hs の変更点}
    変数部分を汎用に(型環境みたいな)
    \begin{verbatim}
type Env = Map String TypeEnv  

data TypeEnv = Integer Integer 
             | TypeBool Bool
             deriving(Show)
    \end{verbatim}
\end{frame}

\begin{frame}[fragile]{Env の変更に伴う評価の変更}
    
    \begin{verbatim}
evalStatement :: Env -> Statement -> Env 

evalExpr :: Env -> Expr -> TypeEnv
    \end{verbatim}
\end{frame}

\begin{frame}[fragile]{デモ③}
\end{frame}

\subsection{}

\begin{frame}{まとめ}
    式を統合し,拡張性を高めた
\end{frame}


\section{今後の展開}
\begin{frame}{今後の展開}
    関数の実装
\end{frame}

\begin{frame}[fragile]{Func の構文的な位置づけ}
    Expr に組み込むことを検討
    \begin{verbatim}
data Expr = ...
          | Func Expr  Statement
          | Apply Expr Expr
    \end{verbatim}
\end{frame}

%
\end{document}

%%% End
